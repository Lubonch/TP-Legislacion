\section{EL SISTEMA DE DERECHOS}

El ordenamiento juridico argentino establece una jerarquia normativa, colocando en lo mas alto a la constitucion nacional junto con los tratados internacionales incorporados mediante el Art. 72 inc. 22

Esta estructura constitucional reconoce la evolcuion historica de los derechos fundamentales, los cuales se pueden categorizar en distintas generaciones segun su desarrollo temporal y caracteristicas.

\begin{enumerate}
    \item \textbf{Derechos de Primera Generación :} Surge durante las grandes revoluciones liberales(como la Revolucion Francesa), con el proposito de de esblecer un limite entre el poder del estao y la autonomia infividual. Incluye la livertad de expresion y el derecho a la intimidad.
    \item \textbf{Derechos de Segunda Generación :} Corresponden a los Derechos Econimicos Sociales y Culturales que sirgieron a principios del siglo XX en respuesta a las desigualdades y explotacion laboral derivadas de la revolucion industrial
    \item \textbf{Derechos de Tercera Generación :} Corresponden a los derechos colectivos, responden a la necesidad de proteger intereses compartidos, como el derecho a un medio ambiente sano, los derechos de los pueblos originarios. Sin estos derechos se pondria en resgo el ejercicio de de las generaciones anteriores.
    \item \textbf{Derechos de Cuarta Generación :} Corresponden a los derechos que nacen del mundo informativo, buca adaptar las garantias fundamentales al entorno digital. Incluyen el derecho al libre acceso a la informacion, la reduccion de la brecha digital y la proteccion de datos personales.
\end{enumerate}   

\section{RIESGOS DIGITALES Y DELITOS INFORMÁTICOS CONTRA MENORES}

La exposicion de los menores de edad a los riesgos del entorno digital requieren la intervencion del derecho penal y civil. En Argentina, el Codigo civil y comercial de la nacion establece el regimen de capacidad progresiva, que afecta como el menor ejerce sus derechos.

\textbf{Capacidad progresiva :} Una persona menor de 13 años no es capaz de ejercer derechos por si misma y requiere la representacion de sus padres o tutores legales. Los adolescente entre 13 y 16 años tiene una capacidad limitada y es escuchado en procesos que los involucran. A partir de los 16 años ya es considerado un adulto para las decisiones relacionadas con el cuidado de su cuerpo.

\subsection{Delitos y contenidos ilicitos}

\begin{enumerate}
    \item \textbf{Pornografia Infantil :} La Ley 26.388 tipifica la pornografia infantil(art 128 Codigo Penal) como la representacion de un menor de dieiocho años dedicado a actividades sexuales explicitas.
    \item \textbf{Grooming :} La Ley 26.904 tipifica el grooming como el acto de un adulto que contacta a un menor de edad mediante medios digitales con el proposito de cometer un delito contra su identidad sexual.
    \item \textbf{Acceso ilegal a datos personales :} El acceso ilegal(Art. 153 bis y 157 bis del Codigo Penal) es tipificado por la Ley 26.388. Incluye la violacion de correos electronicos, redes sociales y otros servicios digitales. 
\end{enumerate}

\section{Regulacion internacional}

\begin{table}[H]
\centering
\begin{tabular}{|l|r|r|r|r|r|r|}
\hline
Jurisdiccion & Ley principal & Edad Minima & Fecha Vigencia & Mecanismo central \\
\hline
Estados Unidos & KOSMA/PKSMA(S.278) & 13 años para la prohibicion de cuentas y 17 años para la prohibicion de algoritmos & Pendiente de promulgacion(Entraria en vigor 1 año despues de su promulgacion) & Prohibe los sistemas de recomendacion personalizados(profiling) para menores de 17 años. \\
Australia & Online Safety Act & 16 años & 10 de Diciembre 2025 & Prohibe el acceso a redes sociales a menores de 16 años. Requiere verificacion de edad a los titulares de cuentas en las plataformas restringidas \\
Union Europera & DS & 13 años (Aprobación PE); debate a 15/16 & 16 de Noviembre de 2022 & Exige a las grandes plataformas online(VLOPs) que eliminen mecanismos que sean adictivos. Busca implementar una solucion general como la "Cartera Digital" para que el control sea mas efectivo \\
\hline
\end{tabular}
\end{table}

\begin{enumerate}
    \item \textbf{PKSMA/KOSMA :} El proyecto S. 278 busca prohibir que los menores de 13 años acedan a redes sociales y limilar los sistemas de recomendacion personalizados(profiling) oara menores de 17 años. 
    \item \textbf{OSA :} La OSA(online safety act) prohibe el acceso a redes sociales a menores de 16 años, exige que las plataformas implementen medidas para el cumplimiento e inpone multas millonarias por la violacion. Exige la verificacion de edad para todos los titulares de cuentas  en plataformas que esten restringidas por edad.
    \item \textbf{DSA :}  La DSA(Digital Service Act) boliga a las grandes plataformas a tener la responsibilidad de proteger a los menores. Pronope la eliminacion de mecanimos edictivos(como las "rachas" o "streaks") y exige verificacion de edad para acceder a servicios en linea.
    El Parlamento Europeo aprobo fijar la edad minima de acceso a redes sociales en 13 años, pero los estados miembros debaten de elevarla a los 15 o 16 años. 
    La comicion europea esta trabajando en el desarrollo de la "European digital wallet" para facilitar la verificacion de la identidad digital y facilitar la implementacion de controles de edad.
/end{enumerate}

\section{El impacto de la reglacion de redes en el sistema de derechos}

La regulacion de las redes sociales imponen deberes de control que generan un impacto directo en las distintas generaciones de derechos.

\begin{enumerate}
    \item \textbf{Libertad de expresion y la primera Generacion :} [Explicar sobre como los filtros y controles podrian afectar la libertad de expresion y van en contra de lo que la primer generacion]
    \item \textbf{Refuerzo de la tercera y cuarta generacion:} [Explicar como la regulacion busca proteger los derechos colectivos y digitales de los menores de edad, reforzando la 3er y 4ta generacion de derechos. Tambien comentar sobre como la verificacion de edad genera insertidumbre sobre como se van a proteger los datos personales(no solo para los menores, los adultos tambien).] 
\end{enumerate}

\section{Como aborda la justicia Argentina la regulacion de redes sociales?}

La justicia argentina aborda los conflictos del entorno digital aplicando los principios generales del derecho adaptandolos a un contexto tecnologico.

\subsection{Responsabilidad de los proveedores de servicios}

Los proveeedores de servicios de internet son definidos como intermediarios tecnologicos que permiten acceder a redes datos, asi como tambien transmitir y almacenar los datos.

La ley 27.078(Ley argentina Digital) declara de interes publico el desarrollo de las TICs y establece principios para los prestadores de serivicios:

\begin{enumerate}
    \item \textbf{Neutralidad de Red :} Garantiza la completa neutralidad de las redes y prohibe que los prestadores de servicios bloqueen, interfieran o restringan el acceso a cualquier contenido o aplicacion, salvo por una orden judicial o una expresa solicitud del usuario.

    \item \textbf{Inviolabilidad de las comunicaciones :} Garantiza la invilabilidad de la correspondencia y comunicaciones electronicas, incluyendo los datos de trafico. Su itnercepcion solo puede realizarse mediante una orden judicial.
\end{enumerate}

Los proveedores de servicio tambien tienen el deber de prevenir el daño(Art 1710 CCyC) y la boligacion de ofrecer software de proteccion y recomendar el uso de programas de bloqueo de sitios pics.[REVISAR!!!]

\subsection{Reponsabilidad parental y el sharenting}

La Responsabilidad parental es el conjunto de deberes y derechos que tiene los padres sobre LA PERSONA Y BIENES DEL HIJo para su proteccion, desarrollo y formacion interal(y dentro del contexto digital tambien incluye la proteccion de su identidad digital y provacidad). El estado actua como garante para exigir el cumplimiento de estos deberes.

La ley 26.061 establece que los menores tienen el derecho a ser respetados en su dignididad, reputacion y propia imagen. 

El sharenting es la practica de los padres donde comparten contenido relacionado con sus hijos en las redes sociales y ha generado jurisprudencia debido a la ausencia de legislacion donde los tribunales intervinieron para proteger los derechos personalisimos de los hijos.

\begin{enumerate}
    \item \textbf{V. F. C/ S. B. S/ MEDIDAS PRECAUTORIAS (ART. 232 DEL CPCC) :} en el fallo el Padre(V. F. C.) solicita que se intime a la madre(S. B.) a abstenerse de subir fotos y videos de sus hijas en redes sociales con fines comerciales.
    El tribunal aplico el interes superior del niño, dandole prioridad a los derechos personalisimos de las niñas sobre el derecho a la livertad de expresion de la madre. 

    El Fallo reafirma que la responsabilidad parental no inluye el derecho a exponer a los hijos en las redes sociales y no puede realizarse de forma unilateral y siempre debe ceder ante el interes y la voluntad del niño.
\end{enumerate}