\section{EL SISTEMA DE DERECHOS}

El ordenamiento juridico argentino establece una jerarquia normativa, colocando en lo mas alto a la constitucion nacional junto con los tratados internacionales incorporados mediante el Art. 72 inc. 22

Esta estructura constitucional reconoce la evolcuion historica de los derechos fundamentales, los cuales se pueden categorizar en distintas generaciones segun su desarrollo temporal y caracteristicas.

\begin{enumerate}
    \item \textbf{Derechos de Primera Generación :}
    \item \textbf{Derechos de Segunda Generación :} 
    \item \textbf{Derechos de Tercera Generación :} 
    \item \textbf{Cuarta Generación :}

\section{RIESGOS DIGITALES Y DELITOS INFORMÁTICOS CONTRA MENORES}

La exposicion de los menores de edad a los riesgos del entorno digital requieren la intervencion del derecho penal y civil. En Argentina, el Codigo civil y comercial de la nacion establece el regimen de capacidad progresiva, que afecta como el menor ejerce sus derechos.

\textbf{Capacidad progresiva :}

\subsection{Delitos y contenidos ilicitos}

\begin{enumerate}
    \item \textbf{Pornografia Infanntil :}
    \item \textbf{Grooming :} 
    \item \textbf{Acceso ilegal a datos personales :} 
    \item \textbf{Datos sensibles :}