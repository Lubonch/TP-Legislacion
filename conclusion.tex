\section*{CONCLUSIÓN}
\addcontentsline{toc}{section}{CONCLUSIÓN}

el avance constante de la tecnologia transformo la forma en la que interactuamos y nos comunicamos entre nosotros. Estos cambios trajeron consigo nuevos desafios y riestos, sobre todo cuando se trata de proteger a los menores de edad. Esto genera la necesidad de establecer marcos regulatorios que los protejan pero que su vez no limiten su libertad de expresion y acceso a la informacion.

La regulacion internacional muestra que no existe una solucion unica, mientras en la union europea se opto por un enfoque que mezcla la regulacion de datos y la tranaparecnaid e los algoritmos, en australia se tomo una medida mas drastica prohibiendo el acceso a las redes sociales para los menores de 16 años. Un gran desafio que tienen estas regulacion  es encontrar el equilobrio en la proteccion de los derechos de primera generacion y la cuarta generacion, ya que la proteccion de los datos busca prevenir que que la dignidad humana se vea afectada por el manejo inscriminado de los datos personales.

En Argentina, el marco juridico actual nos da una base solida que requiere actualizacion que permita abordar estos problemas y la ley de proteccion de datos personales deberia incluir disposiciones especificar sobre algoritmos y verificacion de edad.